\chapter{Resultados obtenidos y trabajos futuros}
\section{Casos de uso}
\subsection{Stacked Borrows y Análisis de cabecera de funciones}

\begin{lstlisting}[
    language=Rust,
    label=code:example1,
    caption={Ejemplo1 mostrado en el paper Stacked Borrows \cite{stackedborrows}}
    ]
fn main() {
    let mut local = 5;
    let raw_pointer = &mut local as *mut i32;
    let result = unsafe { example1(&mut *raw_pointer, &mut *raw_pointer) };
    println!("{}", result); // Prints "13".
}

fn example1(x: &mut i32, y: &mut i32) -> i32 {
    *x = 42;
    *y = 13;
    return *x; // Has to read 42 , because x and y cannot alias !
}
\end{lstlisting}

Al comienzo de la función \textit{main()} podemos apreciar la declaración de una variable mutable(local) y un puntero raw(raw\_pointer)
que hará referencia a la misma. Luego se invoca la función \textit{example1}, la cual toma como argumentos formales 2 referencias mutables,
utilizando la referencia al puntero raw en ambos argumentos. Aquí es donde surgen los problemas. Primero, si bien no esta contemplado por
Stacked Borrows, al llamar una función con más de una referencia mutable indica que la implementación del código puede no ser correcta.
Y segundo, el problema que trata de atacar Stacked Borrows, es el grave problema de aliasing que surge cuando ambos argumentos tratan de
modificar los valores de sus referencias, cuando están apuntando a una misma variable.

Para resolver el primer problema, nuestra herramienta analiza la cabecera de las funciones que son llamadas. Mediante este análisis comprobamos que
no existan dos o más referencias mutables apuntando a una misma variable a la hora de la invocación. Si los argumentos formales incluyen dos o más
referencias mutables se muestra un mensaje "Caution" ya que puede llegar a generar problemas si no se toman las medidas de seguridad necesarias
para la comprobación de los parámetros. Y en el caso de que existan dos o más referencias mutables que hacen referencia a una misma variable, es decir
que exista aliasing, se muestra un mensaje de "Warning" porque es un gran indicativo de que pueden generarse errores debido a esto.
Esto lo podemos apreciar en \href{run:../src/mir_visitor/terminator_visitor.rs}{terminator\_visitor.rs} entre las lineas 33-50

El segundo problema se resuelve mediante la implementación de Stacked Borrows. Generamos una pila donde iremos agregando las variables y referencias
a medida que se van creando, y mortificándola cada vez que se produce una modificación a las mismas. Mediante esta pila simularemos el trabajo del
borrow checker para determinar el tiempo de vida de las variables, extendiéndolo de tal manera que el análisis soporte tanto programas safe como unsafe.
De esta manera, cada vez que la pila se modifica se realiza un chequeo para comprobar si existen referencias mutables X e Y tales que se encuentren
ordenadas "XYXY" en la pila. Esto indicaría una violación de los principios de Stacked Borrows, y por lo tanto, se muestra un "Error". Estos errores
pueden ser que la variable no tenia permisos para escribir o leer, dependiendo del caso.
En el ejemplo, en la función \textit{example1()} podemos ver como romper el principio de la pila genera errores. Queremos retornar 42 en la variable X,
pero como existe aliasing y se realiza una escritura de la forma "XYX", el valor a retornar es diferente. Por lo tanto existe un error ya que la variable
Y no debería haber tenido permitido escribir. Una posible solución seria cambiar los accesos para que queden de forma "YXXY", de esta manera no se rompen
los principios establecidos y el programa seria considerado correcto.

En el siguiente fragmento de la salida de nuestra herramienta podemos apreciar estos mensajes indicativos dentro de la información del MIR.

\begin{lstlisting}[language=Rust]
Main body -- Start

Block bb0 --Start

use bb0[0] Assign _1 = const 5_i32
ref bb0[1] Assign _3 = &mut _1
raw bb0[2] Assign _2 = &raw mut (*_3)
ref bb0[3] Assign _6 = &mut (*_2)
ref bb0[4] Assign _5 = &mut (*_6)
ref bb0[5] Assign _8 = &mut (*_2)
ref bb0[6] Assign _7 = &mut (*_8)
bb0[7] Terminator _4 = example1(move _5, move _7) -> bb1
call example1
Caution: This function call contains two or more mutable arguments
WARNING: Calling function with two mutable arguments that are alias
const ty for<'r, 's> fn(&'r mut i32, &'s mut i32) -> i32 {example1}

Example1 body -- Start

Block bb0 --Start

use bb0[0] Assign (*_1) = const 42_i32
use ERROR Tag <2> does not have WRITE access
bb0[1] Assign (*_2) = const 13_i32
use ERROR Tag <1> does not have WRITE access
bb0[2] Assign _0 = (*_1)
bb0[3] Terminator return

Block bb0 --End
\end{lstlisting}

\subsection{Análisis de cast}

\begin{lstlisting}[
    language=Rust,
    label=code:cast_example,
    caption={Ejemplo de casteo de una variable a otra con diferente representación.}
    ]
pub fn main() {
    let mut one: u64 = 5;
    let raw = &mut one as *mut u64;
    let raw2 = raw as *mut u32;
    unsafe {
        let two = *raw2;
    }
}
\end{lstlisting}

En este ejemplo podemos apreciar como declaramos una variable llamada \textit{one}, que es del tipo u64 y tiene una representación en 8 bytes.
Luego creamos dos punteros raw, donde el primero hace referencia a la primer variable manteniendo su representación de u64. Sin embargo,
en el segundo puntero (raw2), se crea a partir de raw y utilizando como base el tipo u32, que tiene una representación de 4 bytes y es diferente.
En la siguiente instrucción dentro del bloque unsafe hacemos uso del valor dentro de raw2 colocandolo dentro de una variable nueva llamada \textit{two}.\\
Esto es un problema grave, ya que la nueva variable \textit{two} tiene dentro información que probablemente no sea correcta, ya que hicimos un casteo
que no debería ser posible si utilizáramos unicamente referencias dentro de safe Rust. Estamos realizando una transformación de una variable u64
a una u32 que tiene una representación no solo diferente, sino que 4bytes menor, de una manera insegura y el contenido probablemente sea inconsistente
y produzca comportamiento indeseado en el programa.

Nuestra herramienta detecta estas situaciones mediante la comparación del tamaño de la representación de ambas partes de un cast. Si el tamaño es igual
o mayor no hay necesidad de mostrar ningún mensaje de error. Pero cuando queremos hacer una transformación a una representación menor, es muy probable
que suframos de perdida de información; y por este motivo nuestro analizador lo detecta y envía un mensaje informando de la situación. Esto lo podemos
observar programado en el archivo \href{run:../src/mir_visitor/block_visitor.rs}{block\_visitor.rs} entre las lineas 122-150

En el output a continuación se observa el mensaje de Warning asociado a este caso.

\begin{lstlisting}[language=rust]
Main body -- Start
Block bb0 --Start

use bb0[0] Assign _1 = const 5_u64
ref bb0[1] Assign _3 = &mut _1
raw bb0[2] Assign _2 = &raw mut (*_3)
use bb0[3] Assign _5 = _2
kst WARNING: Casting from a layout with 8 bits to 4 bits
from *mut u64 to *mut u32
bb0[4] Assign _4 = move _5 as *mut u32 (Misc)
use bb0[5] Assign _6 = (*_4)
bb0[6] Terminator return

Block bb0 --End
Main body -- End
\end{lstlisting}








